% 输出PDF时,需要把所有路径..替换成.
\documentclass{beamer}
\usepackage[UTF8,noindent]{ctex}
\usetheme{Berkeley}
\usecolortheme{seagull}

% 导言
\title{组会汇报}
\subtitle{移动机器人远程交互软件设计与实现}
\institute{C400}
\author{黎振胜}
\date{\today}
\subject{中南大学研究生学位论文}
\keywords{移动机器人,人机交互,UML,LabVIEW,ROS}

\logo{\includegraphics[width=1.54cm]{../resource/logo/csu.pdf}}
% \logo{\includegraphics[width=1.54cm]{./resource/logo/csu.pdf}}

\begin{document}

\begin{frame}
\maketitle
\end{frame}
%--- Next Frame ---%
\begin{frame}[t]{汇报内容}
    \tableofcontents[pausesections]
\end{frame}
%--- Next Frame ---%
\begin{frame}[t]{1.1背景}
    \section{背景和总体方案}
    \subsection{背景}
    \begin{itemize}
        \item 移动机器人代替人去危险场合执行探测和救援任务的应用背景
            \begin{itemize}
                \item 远场控制;数据可视化;指令传输
                \item 机器人具有一定自主性
            \end{itemize}
        \item 为机器人提供人机交互界面
            \begin{itemize}
                \item LabVIEW软件的优势,具有大量的内置控件
            \end{itemize}
        \item 设备,算法的快速集成
            \begin{itemize}
                \item 部署自然语言交互,决策,规划等算法
                \item 快速连接手柄等设备
            \end{itemize}
    \end{itemize}
\end{frame}
%--- Next Frame ---%
\begin{frame}[t]{1.2总体方案}
    \subsection{总体方案}
\end{frame}
%--- Next Frame ---%
\begin{frame}[t]{2.1使用LabVIEW Actor Framework进行程序设计}
    \section{基于LabVIEW的远程交互平台软件设计和实现}
    \subsection{LVAF}
    \begin{itemize}
        \item 是什么
        \item 为什么
        \item 理解LVAF
    \end{itemize}
\end{frame}
%--- Next Frame ---%
\begin{frame}[t]{一个例子理解LVAF}
    \begin{itemize}
        \item 从循环到并发
        \item 进程间通信
        \item 消息和事件
    \end{itemize}
    配一个样例项目的截图
\end{frame}
%--- Next Frame ---%
\begin{frame}[t]{本软件平台的LVAF组成}
    这里需要一张图
\end{frame}
%--- Next Frame ---%
\begin{frame}[t]{状态设计模式:State Actor}
    讲讲协调者是怎么设计的
\end{frame}
%--- Next Frame ---%
\begin{frame}[t]{看看软件界面}
    这里需要一张图
\end{frame}
%--- Next Frame ---%
\begin{frame}[t]{2.2远程人机交互软件设计的UML描述}
    \subsection{UML}
    \begin{itemize}
        \item UML:面向对象程序的通用建模语言
        \item 七大类图
    \end{itemize}
\end{frame}
%--- Next Frame ---%
\begin{frame}[t]{用例图描述软件设计需求}
    这里需要一张图
\end{frame}
%--- Next Frame ---%
\begin{frame}[t]{组件图描述静态结构}
    这里需要一张图
\end{frame}
%--- Next Frame ---%
\begin{frame}[t]{顺序图描述动态结构}
    这里需要一张图
    并且需要举例子
\end{frame}
%--- Next Frame ---%
\begin{frame}[t]{2.3rosbridge client implementation in LabVIEW}
    subsection{rosbridge client}
    \begin{itemize}
        \item 什么是rosbridge
        \item rosbridge通信规约
        \item 怎么使用
    \end{itemize}
\end{frame}
%--- Next Frame ---%
\begin{frame}[t]{rosbridge client与ROS通信}
    这里需要一张图片
\end{frame}
%--- Next Frame ---%
\begin{frame}[t]{演示}
    \subsection{Video}
    点击此处打开视频
\end{frame}
%--- Next Frame ---%
\begin{frame}[t]{3.基于ROS的移动机器人实验系统设计和实现}
    \section{基于ROS的移动机器人实验系统设计与实现}
\end{frame}
%--- Next Frame ---%

\end{document}